\chapter{Conclusions} 

\section{Final remarks}
The thesis highlights the danger of atrial fibrillation and how there is a need for automatic tools that can detect it and allows to avoid in some cases much more serious diseases. 
Several works in the literature try to define automatic instruments, some are based on the irregularity of the $RR$ segments, others on the rhythm of the heartbeat. The thesis is based on the latter.

Two approaches are proposed to improve the state of the art considered.
\begin{enumerate}
\item The first approach at a global level seeks to improve the number of fibrillating and non-fibrillating beats, correctly classified, by using machine learning techniques to be applied on datasets enriched with morphological features, such as Fourier fast transform and AR coefficients.
In this case, considerable improvements on some records were obtained.
\item The second approach, on the other hand, imposes the objective of finding an optimal threshold that adapts in real-time, working exclusively on the threshold.
The results are promising because they are very close to the optimal threshold, but still need to be improved.
\end{enumerate}

\section{Future works}
Future work will focus on improving individual approaches in a totally independent manner. In the first case, experiment with neural networks. In the second case, try to use genetic algorithms to improve the prediction of local thresholds.
The next step could be to experiment with an adaptive approach on a global model. That is, to adapt the threshold taking into account what is the morphology of the patient.