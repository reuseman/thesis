\chapter{State of the art}
\label{sec:state_of_the_art}

\section{Introduction}
\label{sec:soa_introduction}
In this Chapter, an automatic approach to detect Atrial Fibrillation is analysed. The state of the art is based on different public datasets offered by PhysioNet \cite{physionet}, among which MIT-BIH Atrial Fibrillation Database \cite{afdb} and Long Term AF Database \cite{ltafdb} are used.
The method is based on ECG, whose explanation has been given in Section \ref{sec:ecg_diagnose}. The reason that lies behind the use of ECG, is its intrinsic simplicity, that cannot be found in methods like blood tests, chest x-ray, etc.

\section{Methods of the literature}
Most of the algorithms work on the processing of the ECGs components (P wave, QRS complex, ...) and the poorly coordinate atrial activation (AA) of heart and rapid cardiac beating. Although these pieces of information can lead to the identification of Atrial Fibrillation, noise must be taken into consideration. Especially with P waves which in general is of very low-intensity magnitude. Whereas the approaches based on the RR interval (R wave peak to R wave peak) irregularity, nonetheless the component is a more prominent feature of ECG and thus less subject to noise, tend to be quite complicated and not so efficient to make them suitable for real-time applications \cite[p. 2]{zhou2015}. Examples of noteworthy methods based on RRI are the Petrėnas, et al \cite[2015]{petrenas2015} and Lee, et al \cite[2013]{lee2013}. The former is characterized by the use of ectopic beat filtering, bigeminal suppression and signal fusion, while the latter focus on time-varying coherence functions and Shannon entropy. A real-time and low-complexity algorithm is Zhou et, al \cite{zhou2015}, based on the heart rate. 

\section{The best approach proposed in the literature}
The algorithm Zhou et, al \cite{zhou2015} is a the base of the thesis. Hence a brief explanation of the main characteristics is needed. The method is composed of three steps defined over the heart rate sequence.
\fig{img/zhou.png}{0.8}{zhou_img}{Application of the method to detect AF. (a) is the original sequence $hr_n$; (b) is the symbolic dynamic $sy_n$; (c) the word sequence $wn_n$; (d) the distribution of $\mathcal{H}^{\prime\prime} (\mathbf{A})$.} 
\subsubsection{Heart rate sequence}
Let $hr_n$ be the heartbeat rate sequence obtained from,
\begin{equation}
hr_n = 60\, \text{s} \cdot \frac{250}{R_n - R_{n-1}}
\end{equation}
where $60$ are the seconds, $R_n$ is the sequence that denotes the $R$ peak in the $QRS$ complex and $250$ is the number of samples per second.
\subsubsection{Step 1: symbolic dynamics of $hr_n$ sequence}
Let $sy_n$ denote a symbolic dynamics that encodes the information of $hr_n$ to a series with fewer symbols, where the mapping function is given by \cite[p. 3]{zhou2015},
\begin{equation}
sy_n = \begin{cases}
63 & \text{if $hr_n \ge 315$} \\
\lfloor hr_n / 5 \rfloor & \text{otherwise}
\end{cases}
\end{equation}
where $\lfloor \cdot \rfloor$ is a floor operator. In this way the raw sequence $hr_n$ is transformed in a sequence $sy_n \in [0, 63]$, with 64 instantaneous states \reffigtext{zhou_img}{(b)}.
\subsubsection{Step 2: history sequence of $sy_n$}
A 3-symbols template can be applied to get a window of information that acts as a history \reffigtext{zhou_img}{(c)}, in this case on 3 successive symbols. Through a novel operator-defined below \cite[p. 3]{zhou2015}, the word value can be calculated. 
\begin{equation}
wv_n = (sy_{n-2} \times 2^{12}) + (sy_{n-1} \times 2^{6}) + sy_n
\end{equation}
\subsubsection{Step 3: Shannon entropy}
\label{sec:entropy}
A coarser version of Shannon entropy is employed to discriminate the AF arrhythmias \reffigtext{zhou_img}{(d)}. Without loss of generality, let $\mathbf{A} = (A|P)$ denote a dynamic system. The unique elements in this set can be defined as $A = \{a_1, \ldots, a_k\}$ with the interrelated probability set $P = \{p_1, \ldots, p_k\} (1 \le k \le N)$, where $N$ is the total number of elements and $k$ are the unique elements in space $\mathbf{A}$. Each element $a_i$ has the probability $p_i = N_i/N (0 < p_i \le 1, \sum_{i=1}^{k}p_i=1)$, where $N_i$ is the total number of the specific element $a_i$ in space $\mathbf{A}$. Hence the coarser version of Shannon entropy can be defined to quantitatively calculate the information size of $wv_n$,
\begin{equation}\label{eq:entropy}
\mathcal{H}^{\prime\prime} (\mathbf{A}) = -\frac{k}{N \, log_2 N} \sum_{i=1}^{k}p_i \, log_2 p_i
\end{equation}
The dynamic $\mathcal{A}$ is characterized by a bin size of $N=127$ consecutive word elements from $wv_{n-126}$ to $wv_n$. By defining the characteristic set $A$ and the corresponding probability set $P$, the entropy $\mathcal{H}^{\prime\prime} (\mathbf{A})$ can be calculated. A specific cardiac beat $hr_n$ is labelled as AF if the coarser entropy meets or exceeds a discrimination optimal threshold equal to $0.639$. The threshold was obtained through an investigation of various thresholds in the range $[0.0, 1.0]$ with an increment of $0.001$ from the receiver operating characteristic (ROC) on training databases. 
The computational challenges that are found in the Equation \ref{eq:entropy} can be overcome with a pre-calculated map of $-\frac{1}{log_2 N}p_i \, log_2 pi_i$ \cite[p. 4]{zhou2015}.

\section{Results and comparisons}
The work under consideration measures the performances using sensitivity ($Se$), specificity ($Sp$), positive predictive value ($PPV$), and overall accuracy ($ACC$) \cite[p. 6]{zhou2015}.
\begin{equation}
\label{eq:metrics}
\begin{aligned} S e &=\frac{T P}{T P+F N},\qquad P P V=\frac{T P}{T P+F P} \\ S p &=\frac{T N}{T N+F P}, \qquad A C C=\frac{T P+T N}{T P+T N+F P+F N} \end{aligned}
\end{equation}
where $TP$ stands for true positives, $TN$ true negatives, $FP$ false positives and $FN$ false negatives.
\begin{table}[h]
\begin{threeparttable}
\caption[Performance comparison of some state of the art methods.]{Classification performance of different methods based on three different testing databases \cite[p. 8]{zhou2015}.}
\label{table:zhou_hr_rri}
\scriptsize
  \begin{tabularx}{\linewidth}{c c c c X c c c}
  \toprule
  \textbf{Method} & \textbf{Feature} & \textbf{Year} & \textbf{Database} & \multicolumn{4}{c}{\textbf{Results}} \\
  \cline{5-8}
  \\
  & & & & $\mathbf{SE(\%)}$ & $\mathbf{SP(\%)}$ & $\mathbf{PPV(\%)}$ & $\mathbf{ACC(\%)}$\\
  \midrule  
  Zhou, et al\cite{zhou2015} & HR & 2015 & AFDB & 97.37 & 98.44 & 97.89 & 97.99\\
  & & & AFDB$^1$ & 97.31 & 98.28 & 97.89 & 97.84 \\
  & & & AFDB$^2$ & 98.43 & 98.46 & 97.92 & 98.45 \\
  & & & MITDB & 97.83 & 87.41 & 47.67 & 88.51 \\
  & & & NSRDB & NA & 99.68 & NA & NA \\
  \hline
  Petrênas, et al\cite{petrenas2015} & RRI & 2015 & AFDB & 97.12 & 98.28 & - & -\\
  & & & AFDB$^1$ & 97.1 & 98.1 & - & - \\
  & & & AFDB$^2$ & 98.0 & 98.2 & - & - \\
  & & & MITDB & 97.8 & 86.4 & 47.67 & 88.51 \\
  & & & NSRDB & NA & 98.6 & NA & NA \\
  \hline
  Zhou, et al\cite{zhou2014} & RRI & 2014 & AFDB & 96.89 & 98.25 & 97.62 & 97.67\\
  & & & AFDB$^1$ & 96.82 & 98.06 & 97.61 & 97.50 \\
  & & & AFDB$^2$ & 97.83 & 98.19 & 97.56 & 98.04 \\
  & & & MITDB & 97.33 & 90.78 & 55.29 & 91.46 \\
  & & & NSRDB & NA & 98.28 & NA & NA \\
  \hline
  Lee, et al\cite{lee2013} & RRI & 2014 & AFDB$^2$ & 98.22 & 97.68 & - & 97.91\\
  & & & MITDB & 91.1 & 89.7 & - & - \\
  & & & NSRDB & NA & 99.7 & NA & NA \\
  \bottomrule
\end{tabularx}
\begin{tablenotes}
 	\item $^1$ Records \verb|00735| and \verb|03665| excluded.
	\item $^2$ Records \verb|04936| and \verb|05091| excluded.
	\item ‘NA’ indicates not applicable because there is no beat with AF reference annotation in this database.
    \end{tablenotes}
\end{threeparttable}
\end{table}
\\
A complete overview of the results of the state of the art method explained and others can be found in Table \ref{table:zhou_hr_rri}.
To be sure about experimentation, edge cases are needed. Hence dataset like MITDB which contains many coexisting various types of complex arrhythmias and NSRDB without any AF annotation, are perfect for this purpose. The method performs statistically better than the others \cite[p. 11]{zhou2015} with a very low computational complexity \cite[p. 14]{zhou2015}.

\blankpage