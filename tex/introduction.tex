\chapter{Introduction}
\label{sec:introduction}

\section{What is the Atrial Fibrillation?}
Atrial fibrillation, also abbreviated with AF or A-Fib, is an abnormal heart rhythm that happens when electrical impulses fire off in the atria \reffig{afib_heart}, from different spots without being organized. Characterized by rapid and irregular beating, 	caused by the chambers of the heart twitching \cite{cdcgov17}. This arrhythmia is associated with an increased risk of stroke, heart failure and even dementia \cite{10.1001/archinte.1987.00370090041008, Munger2014}. Affects 5-10 per cent of elderly people.

\fig{img/afib_heart.jpg}{0.6}{afib_heart}{A normal heart beat on the left, an AF heart beat on the right. Image from mayoclinic.org}

The disease is classified by doctors based on how long it lasts or based on the cause. The treatment will be different for each kind \cite{webmd2018}:
\begin{itemize}
  \item Paroxysmal Atrial Fibrillation (holiday heart syndrome): an episode of AF, the duration of whose may be a few minutes or a few days, but which tends to be below the week. Usually, treatment is not needed;
  \item Persistent: the disease lasts longer than a week and it can stop on its own, or a specific medicine or treatment is needed. If the latter does not work, doctors opt for the electrical cardioversion, which is a low-voltage current used to reset the normal rhythm;
  \item Permanent: also called chronic, cannot be treated. The doctor decides for a long term medication to reduce the odds of associated health conditions.
\end{itemize}

\section{Causes and symptoms}
There are many possible causes of the condition, some are controllable, others are not. Cardiovascular factors play a big role: high blood pressure, heart valve disease, congenital heart disease and even previous heart surgery. But difficulties in breathing are a key factor too, in other words obesity, obstructive sleep apnea \cite{doi:10.1111/obr.12056}. Alcohol consumption and tobacco smoking are associated with an increased risk of developing atrial fibrillation \cite{Tonelo2013, DU20171968}. Other factors are genetics, ageing, a sedentary lifestyle and diabetes \cite{10.1001/jama.291.23.2851, Staerk2017}.

The person often feels an abnormal beating that starts to become longer and constant. There could be heart palpitations, shortness of breath, chest pain,  light-headedness, or fainting \cite{chamberlain_gray_houghton_2010}. But the biggest problem is that often these kind of episodes are asymptomatic \cite{Munger2014}, in fact sometimes first diagnosed when patients present a stroke \cite{page2003asymptomatic}.



causa stroke 20%, rischio ictus 5x

DIAGNOSI: complessa quindi anche più esami
- Elettrocardiogramma (ECG)
- Prova da sforzo
- Dispositivi di monitoraggio a lungo termine
        Registratore di eventi
        Holter
        Monitor cardiaco impiantabile