\section*{\hfil Abstract\hfil}

\begin{center}
\vspace*{\stretch{1}}
Atrial fibrillation (AF) is one of the most common cardiac arrhythmia associated with an increase risk of stroke, heart failure and dementia. The estimated number of individuals with the disease is around 33 million, which is a downward forecast as the global population ages. In addition, the asymptomatic characteristic makes this arrhythmia even more severe.
\newline

The thesis describe the state of the art algorithm based on heart rate for the automatic detection of Atrial fibrillation through the use of an Electrocardiogram (ECG).
The explained algorithm was used as a basis for achieving better performance by using Machine Learning techniques and algorithms globally on available medical databases. A local experimentation was then carried out, which takes into account the different physiologies of the patients and consequently groups them according to optimal discriminatory thresholds.
\newline

On the global level significant improvement were made. Instead the local level turned out to be more complicated than expected and simple improvements through a generalized least squares model couldn't be achieved. To conclude, future developments aim to create a better replica of the state of the art and carry out a comprehensive experimentation with neural networks on a global level. As for the local one, more complex regression models will have to be used to group patients together.
\vspace*{\stretch{1}}
\end{center}
\newpage